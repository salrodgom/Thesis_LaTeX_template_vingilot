% This is part of a Thesis Template by Salvador Rodríguez-Gómez.
% This template is inspired in part by several other templates.
% It is distributed under the Creative Commons Attribution-ShareAlike license.
% http://creativecommons.org/licenses/by-sa/3.0/
%
\cleardoublepage
\phantomsection
\addcontentsline{toc}{chapter}{Conclusions}
\setcounter{footnote}{0}
\chapter*{\textbf{Conclusions}}
The \textbf{main finding} in this thesis is that it is possible to obtain molecular insights into flexibility of soft nanoporous crystals by descending to atomic level. As we assume in the hypothesis, there is strong interplay between structural changes and sorption and transport properties on nanoporous crystals. Molecular simulation can be an extremely useful tool: calculating and predicting observables which can be compared with experimental findings, to obtain knowledge on the nature of the flexibility of these materials.

The Al$^{+3}$/Si$^{4+}$ cation distribution has minor impact on the phase transition associated to the flexibility of zeolite RHO. Therefore, pure silica structure provides an initial model for the study of the phase transition between acentric and centric space groups. The role of point charges and atom polarisability is crucial to describe the structural distortion, the volume, cell size and cell shape. The main conclusion in Chapter \ref{chap_RHO1} is that interatomic potentials from molecular mechanics (those that involve an appreciable sort of interactions for bonds, bendings, and torsion) are unable to correctly reproduce the structure because they try to linearise a strictly energetic non-linear problem. The use of a polarisable force field and anharmonic potentials in Chapter \ref{chap_RHO2}, like the shell model for instance that stabilises low-symmetry structures, is mandatory for this purpose.

A novel method, based on cycles of the combined use of Monte Carlo, Energy Minimisation and Molecular Dynamics methods, has been proposed for the study of structural changes of high flexible nanoporous materials that are associated to the effect of temperature or guest molecules. The method provides crystallographic-quality structures (0.07-0.2\% deviation from experimental values) of zeolite RHO exchanged with different extra-framework cations and different water content. A close relation was found between the polarising power of the extra-framework cations and the effective pore windows. The amount of water also modulates the effective pore windows. By choosing the right combination of extra-framework cations and water content, one can design the size of the effective pore windows for  targeted molecular separations.

The \VIR{\ce{Ge}$^{4+}$/\ce{Si}$^{4+}$ cation distribution has a huge impact on the associated flexibility and stability of STW-type germanosilicate.
This is a consequence of the difference in the deformability and size of silicon and germanium tetrahedra.
An effective Hamiltonian has been performed to successfully study the colossal number of configurations in the whole compositional range (Ge$_f$ = Ge/(Ge+Si)= 0 to 1). 

\selectlanguage{spanish}
\phantomsection
\addcontentsline{toc}{chapter}{Conclusiones}
\setcounter{footnote}{0}
\chapter*{\textbf{Conclusiones}}
%\vskip-5mm
La \textbf{principal conclusión} de esta tesis es que es posible obtener una visión molecular de la flexibilidad de cristales blandos nanoporosos al descender a un nivel atómico. Como suponemos en la hipótesis, existe una fuerte interacción entre los cambios estructurales y las propiedades de adsorción y transporte en cristales nanoporosos: cada fenómeno actúa retroalimentando, positiva o negativamente, a los otros fenómenos. La simulación molecular ha sido una herramienta extremadamente útil de esta manera: el cálculo y la predicción de observables contrastables con los hallazgos experimentales, con el fin de obtener un conocimiento más profundo de la naturaleza de la flexibilidad de estos materiales.

La distribución de Si/Al tiene un impacto menor en la transición de fase asociada a la flexibilidad de la zeolita RHO de lo supuesto inicialmente. Por tanto, el material pura sílice proporciona un modelo inicial para el estudio de la transición de fase entre los grupos espaciales acéntricos y céntricos. El papel de las cargas y la polarizabilidad del átomo es crucial para describir estas distorsiones estructurales, cambios de volumen, y los tamaños y formas de la celda unidad. La principal conclusión en el Capítulo \ref{chap_RHO1} es que los potenciales interatómicos que provienen de la mecánica molecular (aquellos que implican un número apreciable de interacciones para enlaces, flexiones y torsiones) no pueden reproducir correctamente la estructura porque intentan linealizar un problema energético estrictamente no lineal. El uso de un potencial interatómico polarizable en el Capítulo \ref{chap_RHO2}, como el modelo de núcleo--corteza, estabiliza las estructuras de baja simetría y es obligatorio para este tipo de cálculos.

Se ha propuesto un método basado en ciclos combinados de métodos Monte Carlo, minimizaciones energéticas y dinámicas moleculares para el estudio de los cambios estructurales que ocurren en materiales nanoporosos altamente flexibles asociados al efecto de la temperatura o adsorbatos. Este método proporciona estructuras de alta calidad cristalográfica (desviaciones del 0.07-0.2\% respecto a valores experimentales) de la zeolita tipo RHO intercambiada con diferentes cationes libres y diferente contenido de agua. Se encontró una relación entre el poder de polarización de los cationes libre y la apertura de las ventanas del poro. La cantidad de agua también modula la apertura de las ventanas. Eligiendo una combinación específica de cationes libres y contenido de agua, es posible diseñar y modular la apertura de estas ventanas para una separación molecular específica.

La distribución de cationes \ce{Ge}$^{4+}$/\ce{Si}$^{4+}$ tiene un impacto importante en la estabilidad y flexibilidad asociada de las zeolitas tipo STW.
Esto es una consecuencia de la diferente deformabilidad y tamaño de los tetraedros de germanio y silicio.
Un hamiltoniano efectivo fue desarrollado y diseñado con éxito para estudiar el enorme número de configuraciones en todo el intervalo de fracciones molares (Ge$_f$ = Ge/(Ge+Si)= de 0 a 1).
A medida que la fracción molar de Ge aumenta, la energía libre desciende a un mínimo y luego aumenta súbitamente hasta la composición pura de Ge. 
Nuestro estudio de modelización ha mostrado la compleja presencia de múltiples mínimos poco profundos en la hipersuperficie de energía, lo cual es una explicación de la flexibilidad estructural de las zeolitas que contienen germanio. 

\selectlanguage{british}
