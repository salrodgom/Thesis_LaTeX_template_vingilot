% This is part of a Thesis Template by Salvador Rodríguez-Gómez.
%This template is inspired in part by several other templates.
% It is distributed under the Creative Commons Attribution-ShareAlike license.
% http://creativecommons.org/licenses/by-sa/3.0/
%
\chapter{Structural distortions by cosa migration and cosation:\\ LOL--type zeolite I}
\label{chap_RHO2}
Molecular valves are nanostructured materials that are becoming popular, due to their potential use in bio-medical applications. However, little is known concerning their performance when dealing with small molecules, which are of interest in energy and environmental areas. It has been observed experimentally that zeolite RHO shows unique pore deformations upon changes in hydration, cation siting, cation type, and/or temperature-pressure conditions. By varying the level of distortion of double 8-rings it is possible to control the adsorption properties, which confers a molecular valve behaviour to this material. We have employed interatomic potentials-based simulations to obtain a detailed atomistic view of the structural distortion mechanisms of zeolite RHO, in contrast with the averaged and space group restricted information that can be retrieved from diffraction studies. We have modeled the pure silica zeolite RHO as well as four aluminosilicate structures, containing Li$^+$, Na$^+$, K$^+$, Ca$^{2+}$ and Sr$^{2+}$ cations. It has been found that the distortions of the three zeolite rings are coupled, although the four-membered rings are rather rigid and both six- and eight-membered rings are largely flexible. A large dependence on the polarising power of the extra-framework cations and with the loading of water has been found for the minimum aperture of the eight-membered rings that control the nanovalve effect. The energy barriers needed to move the cations across the eight-membered rings are calculated to be very high, which explains the origin of the experimentally observed slow kinetics of the phase transition, as well as the appearance of metastable phases.


The publication related with this section can be found in:
\begin{small}
\begin{itemize}
\item \fullcite{Balestra2015}.
\end{itemize}
\end{small}

\section{Introduction}
Molecular valves are a class of molecular devices that allow molecular transport in a controlled way through gate opening or trapdoor mechanisms. Valve action is typically performed by a molecule that is attached to the material, either by covalent bonds, hydrogen bonds or supramolecular interaction. In presence of external stimuli, such as temperature, pressure, pH, molecular or ion chemical potential, this molecule is able to change its configuration to allow the molecular flow. This ability has attracted huge attention during the last years, due to its impact in delivering medium and large size active molecules for medical applications \cite{mengvalve2010,C1MD00158B,doi:10.1021/nn3018365,doi:10.1021/nn101499d}.
\begin{figure}[!htpb]
  \centering
  \includegraphics[width=0.55\textwidth]{./images/i43m-surface-view.eps}
  \caption{\label{fig:i43m}Snapshot of a distorted form of RHO-type zeolite obtained by Molecular Dynamics. An isoenergy surface is shown in translucent gray. Extra-framework cations are omitted for clarity.
  }
\end{figure}

\lipsum[1-12]
