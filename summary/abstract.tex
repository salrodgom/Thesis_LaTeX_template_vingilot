% This is part of a Thesis Template by Salvador Rodriguez-Gomez.
% This template is inspired in part by several other templates.
% It is distributed under the Creative Commons Attribution-ShareAlike license.
% http://creativecommons.org/licenses/by-sa/3.0/

\phantomsection
\addcontentsline{toc}{chapter}{Summary}
\setcounter{footnote}{0}

\chapter*{\textbf{Summary}}
This thesis is focused on the study of nanoporous soft crystals. Atoms in crystalline material in general fluctuate around their equilibrium \VIR{positions}. External stimuli and thermal effects distort the atoms, possibly resulting in a loss of crystallinity. The main \textbf{hypothesis} of the dissertation that I present is: \emph{a)} the lack of crystalline regularity has an impact on the microscopic porosity, the void-channels and pockets available to guest molecules, and an effect in the sorption- and transport-properties, but likewise \emph{b)} adsorbates could affect the pore structure, in a bidirectional causal relation, and \emph{c)} these phenomena can be modelled with appropriate simulation techniques.


The aim and scope of the thesis is twofold:
\begin{enumerate}
\item To study thermal--induced and guest--induced distortions in soft nanoporous materials by means of computer simulation.
\item To develop new computational algorithms which improve the efficiency of simulations and ensure convergence to true equilibrium.
\end{enumerate}


This thesis is structured as follows:\\

Chapter \ref{chap_intro} covers \ldots

The final part is dedicated to the conclusions as well as a brief discussion on the perspective this work can have for future work-directions.


\phantomsection
\addcontentsline{toc}{chapter}{Resumen}
\setcounter{footnote}{0}

\chapter*{\textbf{Resumen}}
Esta tesis está orientada al estudio de cristales nanoporosos flexibles. Como se verá a lo largo del desarrollo del trabajo, algunas consecuencias y consideraciones emergen al tener en cuenta que los átomos del cristal pueden moverse más allá de los límites que marcan las fluctaciones térmicas. A esto hay que añadir una posible pérdida de la regularidad cristalina. La principal \textbf{hipótesis} de este trabajo de tesis es: \textit{a)} la disminución de regularidad cristalina tiene un impacto en la porosidad microscópica accesible, en general, a adsorbatos, y un efecto en las propiedades de adsorción y transporte, pero a su vez \emph{b)} los adsorbatos pueden afectar la estructura del poro en un proceso de retroalimentación causal, \emph{c)} siendo estos fenómenos modelables usando las técnicas apropiadas de simulación.


Los objetivos de esta tesis son:
\begin{enumerate}
\item Estudiar las distorsiones inducidas por la temperatura y por moléculas adsorbidas en materiales nanoporosos flexibles mediante simulación por ordenador.
\item Desarrollar algoritmos computacionales que mejoren la eficiencia de estas simulaciones y que aseguren una convergencia real a estados de equilibrio.
\end{enumerate}


La tesis se estructura de la siguiente manera:\\

El Capítulo \ref{chap_intro} está dedicado a \ldots

La parte final de la tesis está dedicada a las conclusiones y discusión breve de la proyección que puede llegar a tener este trabajo en el futuro.
